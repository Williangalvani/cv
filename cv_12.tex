%%%%%%%%%%%%%
% Deedy CV/Resume
% XeLaTeX Template
% Version 1.0 (5/5/2014)
%
% This template has been downloaded from:
% http://www.LaTeXTemplates.com
%
% Original author:
% Debarghya Das (http://www.debarghyadas.com)
% With extensive modifications by:
% Vel (vel@latextemplates.com)
%
% License:
% CC BY-NC-SA 3.0 (http://creativecommons.org/licenses/by-nc-sa/3.0/)
%
% Important notes:
% This template needs to be compiled with XeLaTeX.
%
%%%%%%%%%%%%%%%%%%%%%%%%%%%%%%%%%%%%%%

\documentclass[letterpaper]{deedy-resume} % Use US Letter paper, change to a4paper for A4 

\begin{document}

%----------------------------------------------------------------------------------------
%	TITLE SECTION
%----------------------------------------------------------------------------------------

\lastupdated % Print the Last Updated text at the top right

\namesection{Willian}{Galvani}{ % Your name
\urlstyle{same}\url{} \\ % Your website, LinkedIn profile or other web address
\href{mailto:williangalvani@gmail.com}{williangalvani@gmail.com} | (48)9996-5310 % Your contact information
}



%----------------------------------------------------------------------------------------
%	LEFT COLUMN
%----------------------------------------------------------------------------------------

\begin{minipage}[t]{0.33\textwidth} % The left column takes up 33% of the text width of the page

%------------------------------------------------
% Education
%------------------------------------------------

\section{Educação} 

\subsection{UFSC}
\descript{Bacharelado em Ciências da Computação}
\location{Incompleto: 2008-2010}

\descript{Engenharia de Controle e Automação}
\location{7ª fase, 2011 - Atualmente}


\sectionspace % Some whitespace after the section

%------------------------------------------------
% Coursework
%------------------------------------------------

% \section{Coursework}

% \subsection{Physics}
%  Waves and Optics, Electromagnetism I and II, Intro to Quantum Mechanics I and II, Statistical Thermodynamics, Modern Physics, Classical Mechanics

% \subsection{Mathematics}
% Mathematics of Signal Processing, Linear Algebra I and II, Probability and Statistics I, Applied Math I and II, Complex Variables, Partial Differential Equations, Discrete Structures\\

% \sectionspace % Some whitespace after the section

\section{Habilidades}


\location{Programação}
Python \textbullet{} C  \textbullet{} C++ \textbullet{} C\# \textbullet{} \LaTeX\ \\ \textbullet{} HTML\textbullet{} Javascript \\
\location{Outros}
\textbullet CAD(SolidWorks/Inventor) \textbullet Linux \textbullet Windows \textbullet Django \textbullet Arduino \textbullet Proteus \textbullet Eagle


\sectionspace % Some whitespace after the section

%\section{Awards}
%Arthur Zamanakos Scholarship (excellence in Mathematics) \textbullet{} Ye-Yung Teng Memorial Scholarship (excellence in Physics) \textbullet{} Kennedy Family Merit Scholarship (June 2012, June 2013)

\section{Interesses}
Aeromodelismo, Programação,\\ Robótica móvel, Veículos Autônomos,\\Visão Computacional,\\ Inteligência Artificial.

\section{Línguas}
Português(nativo),\\ Inglês(Toefl IBT 106/120).

\section{GitHub}
\href{https://github.com/Williangalvani}{GitHub.com/Williangalvani}

\section{Website}
\href{https://GalvanicLoop.com}{GalvanicLoop.com}

%----------------------------------------------------------------------------------------

\end{minipage} % The end of the left column
\hfill
%
%----------------------------------------------------------------------------------------
%	RIGHT COLUMN
%----------------------------------------------------------------------------------------
%
\begin{minipage}[t]{0.66\textwidth} % The right column takes up 66% of the text width of the page



%------------------------------------------------
% Experience
%------------------------------------------------

\section{Experiência Profissional}

\runsubsection{UFSC - Departamento de Automação e Sistemas}
\descript{Desenvolvimento e manutenção de websites}
\location{Agosto 2011 – Dez 2013}
\vspace{\topsep} % Hacky fix for awkward extra vertical space
\begin{tightitemize}
 \item Desenvolvimento e manutenção dos websites do Departamento e do curso de Eng. de Controle e automação, utilizando o framework Django, em Python.
\end{tightitemize}

\sectionspace % Some whitespace after the section

\descript{Desenvolvimento de sistema de controle de acesso}
\location{Novembro 2013 – Junho 2014}
\vspace{\topsep} % Hacky fix for awkward extra vertical space
\begin{tightitemize}
 \item Desenvolvimento e manutenção de um sistema de gerenciamento do controle de acesso interno do departamento, interfaceando com os sistema da Automatiza, utilizando o framework Django, em Python.
\end{tightitemize}

\sectionspace % Some whitespace after the section


\runsubsection{UFSC - Laboratorio Controle de automação}
\descript{\\ ProVant- Projeto de Veiculo Aéreo não-tripulado.}
\location{Julho 2012 – Dez 2014}
\vspace{\topsep} % Hacky fix for awkward extra vertical space
\begin{tightitemize}
 \item Piloto e Desenvolvedor no projeto ProVant de uma aeronave Tilt-Rotor autônoma.
\end{tightitemize}

\sectionspace % Some whitespace after the section

%------------------------------------------------


%------------------------------------------------
% Research
%------------------------------------------------

\section{Conhecimentos Relevantes }

\runsubsection{Aeromodelismo}
\location{ 2011 – Atualmente}
\begin{tightitemize}
%\vspace{\topsep} % Hacky fix for awkward extra vertical space
\item Noções básicas de aerodinâica e funcionamento de aeronaves asa baixa, asa alta, e multirotores . 
\item Noções de uso, manutenção e dimensionamento de baterias li-po.
\item Noções de funcionamento de sistemas autonomos de navegação e controle de aeronaves autônomas.
\end{tightitemize}

\sectionspace % Some whitespace after the section

%------------------------------------------------

%\runsubsection{Explorations in Quasi-Randomness}
%\descript{| Undergraduate Researcher and Co-Founder}
%\location{Oct. 2012 – Dec. 2013 | Lowell, MA} 
%\begin{tightitemize}
%\vspace{\topsep}
%\item Initiated the start up of the group with \textbf{\href{http://faculty.uml.edu/jpropp/}{Prof. James Propp}} and recruited students to join. 
%\item Worked in a small group on combinatorics problems related to the de-randomized Polya and Smirnov boards. 
%\item Presented work at an undergraduate mathematics conference.
%\end{tightitemize}

\sectionspace % Some whitespace after the section

\section{Projetos Relevantes}

\runsubsection{Asa Voadora}
%\descript{| Fall 2012}
\location{    }
\begin{tightitemize}
\item Projeto do corpo baseado no perfil Zagui12.
\item Corpo em isopor P3, esculpido com a técnica de hot-wire.
\item Atualmente sendo utilizada para dar suporte a uma pesquisa sobre o uso de energia solar em aeronaves de asa fixa.
\end{tightitemize}

\sectionspace

\runsubsection{Multi-rotores}
%\descript{| Spring 2013}
\location{   }
\begin{tightitemize}
   \item Construção de vários multi-rotores, baseados no software MultiWii, com controladora DIY.
\item Atualmente com um Quadricóptero e um Tricóptero em funcionamento.   
\item Atualmente projetando um micro-quadcóptero em placa de circuito.

%\item Analyzed the collected data to determine  properties of the electronic device or circuit. 
\end{tightitemize}

\sectionspace

\runsubsection{FPV}
%\descript{| Fall 2013}
\location{    }
\begin{tightitemize}
\item \href{https://github.com/Williangalvani/RpiMultiwiiFpv}{RpiMultiwiiFpv}: Sistema de Fpv baseado em Raspberry PI, WiFi, e Multiwii. 

\item \href{https://github.com/Williangalvani/diyosd\_multiwii}{DIYOSD\_MultiWii}: OSD baseado em arduino e alguns componentes passivos, com dados obtidos pelo Multiwii.

\item \href{https://github.com/patrickelectric/provant-groundstation}{provant-groundstation}: Software de estação base para o projeto ProVant.

\item \href{https://github.com/Williangalvani/groundstation\_pi3d}{Groundstation\_Pi3D}: Groundstation para Raspberry pi, utilizando aceleração 3D para renderização.

%\item Implemented LabVIEW software by setting up the experimental hardware to conduct a computer-controlled experiment that would acquire and display data.
\end{tightitemize}


%----------------------------------------------------------------------------------------

\end{minipage} % The end of the right column

%----------------------------------------------------------------------------------------
%	SECOND PAGE (EXAMPLE)
%----------------------------------------------------------------------------------------

%\newpage % Start a new page

%\begin{minipage}[t]{0.33\textwidth} % The left column takes up 33% of the text width of the page

%\section{Example Section}

%\end{minipage} % The end of the left column
%\hfill
%\begin{minipage}[t]{0.66\textwidth} % The right column takes up 66% of the text width of the page

%\section{Example Section 2}

%\end{minipage} % The end of the right column

%----------------------------------------------------------------------------------------

\end{document}