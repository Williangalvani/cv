%%%%%%%%%%%%%
% Deedy CV/Resume
% XeLaTeX Template
% Version 1.0 (5/5/2014)
%
% This template has been downloaded from:
% http://www.LaTeXTemplates.com
%
% Original author:
% Debarghya Das (http://www.debarghyadas.com)
% With extensive modifications by:
% Vel (vel@latextemplates.com)
%
% License:
% CC BY-NC-SA 3.0 (http://creativecommons.org/licenses/by-nc-sa/3.0/)
%
% Important notes:
% This template needs to be compiled with XeLaTeX.
%
%%%%%%%%%%%%%%%%%%%%%%%%%%%%%%%%%%%%%%

\documentclass[A4]{deedy-resume} % Use US Letter paper, change to a4paper for A4 
\usepackage[portuguese]{babel}
\usepackage{fontspec}
\usepackage{fontawesome}
\usepackage{array}
\newcolumntype{y}[1]{>{\let\newline\\\arraybackslash\hspace{0pt}}p{#1}}
\renewcommand*{\arraystretch}{1.2}

\usepackage{tikz}
\newcommand\resume[2]{%
  \ifnum#1>#2
    $#1 > #2$
  \else
    \ifnum#1<0
      $#1 < 0$
    \else
      \ifnum#2<0
        $#2 < 0$
      \else
        \tikz{%
        \ifx#20
        \else
          \foreach \i in {1,...,#2} {
            \filldraw[black!20] (\i ex,0) circle (0.4ex);
          };
        \fi
        \ifx#10
        \else
          \foreach \i in {1,...,#1} {
            \filldraw[black] (\i ex,0) circle (0.4ex);
          };
        \fi
        }
      \fi
    \fi
  \fi
}

\begin{document}

%----------------------------------------------------------------------------------------
%	TITLE SECTION
%----------------------------------------------------------------------------------------

%\lastupdated % Print the Last Updated text at the top right

\namesection{Willian}{Galvani}{ % Your name
% Your website, LinkedIn profile or other web address
\href[pdfnewwindow=true]{mailto:williangalvani@gmail.com}{\faEnvelopeO \, williangalvani@gmail.com} \,  \faPhone \, (48)99996-5310 % Your contact information
}



%----------------------------------------------------------------------------------------
%	LEFT COLUMN
%----------------------------------------------------------------------------------------

\begin{minipage}[t]{0.33\textwidth} % The left column takes up 33% of the text width of the page

%------------------------------------------------
% Education
%------------------------------------------------

\section{Educação} 

\subsection{UFSC}
\descript{Bacharelado em Ciências da Computação}
\location{Incompleto: 2008-2010}
\vspace*{0.2cm}
\descript{Engenharia de Controle e Automação}
\location{2011 - 2017}
\vspace*{0.2cm}
\subsection{SEAMK}
\descript{(Intercâmbio em Automation Engineering)}
\location{2014.2-2015.2}

\sectionspace % Some whitespace after the section

%------------------------------------------------
% Coursework
%------------------------------------------------

% \section{Coursework}

% \subsection{Physics}
%  Waves and Optics, Electromagnetism I and II, Intro to Quantum Mechanics I and II, Statistical Thermodynamics, Modern Physics, Classical Mechanics

% \subsection{Mathematics}
% Mathematics of Signal Processing, Linear Algebra I and II, Probability and Statistics I, Applied Math I and II, Complex Variables, Partial Differential Equations, Discrete Structures\\

% \sectionspace % Some whitespace after the section

\section{Habilidades}


\location{Programação}
\begin{tabular}{y{55pt}y{65pt}}
Python  & \resume{7}{10} \\ 
C & \resume{6}{10}  \\
C++ & \resume{5}{10} \\
C\# & \resume{5}{10} \\
\LaTeX\ & \resume{5}{10} \\
MATLAB & \resume{5}{10} \\
HTML & \resume{6}{10}\\
JavaScript & \resume{6}{10} \\
CSS & \resume{4}{10}\\
DRF & \resume{3}{10}
\end{tabular}\\

\vspace*{0.2cm}

\location{Frameworks/Bibliotecas}
\begin{tabular}{y{55pt}y{65pt}}
Arduino & \resume{7}{10} \\
Django & \resume{7}{10} \\
Ionic & \resume{5}{10} \\
OpenCV &\resume{5}{10} \\
Qt &\resume{4}{10}
\end{tabular}
\vspace*{0.2cm}

\location{Outros}
\begin{tabular}{y{55pt}y{65pt}}
GIT & \resume{7}{10} \\
Inventor & \resume{6}{10} \\
Linux & \resume{6}{10} \\
Redes & \resume{6}{10} \\
SolidWorks & \resume{5}{10} \\
Proteus & \resume{5}{10} \\
V-REP & \resume{5}{10} \\
Eagle & \resume{4}{10} \\
\end{tabular}

\vspace*{0.3cm}
Escala:\\
\scriptsize 1-Conhecimento superficial.\\
2-Habilidade de fazer pequenas modificações.\\
3,4-Capaz de desenvolver pequenos projetos.\\
5,6-Confortável com a ferramenta, capaz de concretizar projetos mais complexos.\\
7,8-Conhece a até o funcionamento interno da ferramenta. \\
9,10-Conhecimento abrangente, de toda a ferramenta, implementações, e nuances. Capaz de recriá-la dado o tempo necessário.

\sectionspace % Some whitespace after the section



%\section{Awards}
%Arthur Zamanakos Scholarship (excellence in Mathematics) \textbullet{} Ye-Yung Teng Memorial Scholarship (excellence in Physics) \textbullet{} Kennedy Family Merit Scholarship (June 2012, June 2013)


%----------------------------------------------------------------------------------------

\end{minipage} % The end of the left column
\hfill
%
%----------------------------------------------------------------------------------------
%	RIGHT COLUMN
%----------------------------------------------------------------------------------------
%
\begin{minipage}[t]{0.66\textwidth} % The right column takes up 66% of the text width of the page



%------------------------------------------------
% Experience
%------------------------------------------------
\section{Experiência Profissional}

\runsubsection{Novarum Sky}\\
\descript{Full Stack Embedded Software Engineer.}
\descript{(Python + Raspberry Pi + Gstreamer + Asterisk + VUE.JS + HTML + CSS)}
\location{March 2017 – Atualmente}
\vspace{\topsep}
\begin{tightitemize}
 \item Desenvolvimento de sistemas auxiliares para embarque em drones para transmissão de audio e vídeo HD a longas distâncias.
 \item Desenvolvimento de um sistema de inspeção industrial com drones.
 \item Projeto e prototipação de uma aeronave de asa fixa capaz de pouso e decolagem verticais para foogrametria (Projeto de Fim de Curso).
\end{tightitemize}
\sectionspace % Some whitespace after the section

\runsubsection{UFSC / FEESC / Petrobras}
\descript{\\ Otimização sem derivadas para a sintonia automática de um simulador de poços de petróleo.}
\descript{(Python + Matplotlib)}
\location{Agosto 2016 – Agosto de 2017}
\begin{tightitemize}
 \item Utilização de otimizadores para funções caixa-preta.
 \item Implementação de métodos de otimização sem derivadas em Python.
 \item Implementação de Interface de otimizadores com um simulador interno da Petrobras.

\end{tightitemize}

\sectionspace % Some whitespace after the section

\runsubsection{Instituto SESI de inovação}
\descript{\\ Desenvolvimento de softwares embarcados.}
\descript{\\(C++ + Kinetis + Python + Asterisk)}
\location{Agosto 2015 – Fevereiro 2016}

\begin{tightitemize}
 \item Desenvolvimento em Python para Raspberry PI.
 \item Desenvolvimento em C para plataforma freedom K64f.
\end{tightitemize}



\vspace{\topsep} % Hacky fix for awkward extra vertical space


\runsubsection{UFSC - Laboratório de Controle de automação}
\descript{\\ ProVant- Projeto de Veículo Aéreo não-tripulado.}
\descript{\\(Python + Qt + Arduino)}
\location{Julho 2012 – Junho 2014}

\begin{tightitemize}
 \item Piloto e Desenvolvedor no projeto ProVant de uma aeronave Tilt-Rotor autônoma.
 \item Projeto de protocolo de comunicação entre VANT e estação-base.
 \item Projeto elétrico, eletrônico, e mecânico de VANTs.
 \item Desenvolvimento de sofware para estação base em Python e Qt.
% * <patrickelectric@gmail.com> 2016-12-23T15:58:56.004Z:
%
% Fala dos paranaue que nos fez.
% interface gráfica, comunicação montagem eletronica e essas coisas
%
% ^.
\end{tightitemize}

\sectionspace % Some whitespace after the section



\runsubsection{UFSC - Departamento de Automação e Sistemas}
\descript{Desenvolvimento e manutenção de websites}
\descript{\\(Python + Django + JS + CSS + HTML)}
\location{Agosto 2011 – Dez 2013}

\begin{tightitemize}
 \item Desenvolvimento e manutenção dos websites do Departamento e do curso de Eng. de Controle e automação, utilizando o framework Django, em Python.
\end{tightitemize}

\sectionspace % Some whitespace after the section

\descript{Desenvolvimento de sistema de controle de acesso}
\location{Novembro 2013 – Junho 2014}
\vspace{\topsep} % Hacky fix for awkward extra vertical space
\begin{tightitemize}
 \item Desenvolvimento e manutenção de um sistema de gerenciamento do controle de acesso interno do departamento, interfaceando com os sistema da Automatiza, utilizando o framework Django, em Python.
\end{tightitemize}

\sectionspace % Some whitespace after the section



%------------------------------------------------


%------------------------------------------------
% Research
%------------------------------------------------

\sectionspace % Some whitespace after the section

%------------------------------------------------

%\runsubsection{Explorations in Quasi-Randomness}
%\descript{| Undergraduate Researcher and Co-Founder}
%\location{Oct. 2012 – Dec. 2013 | Lowell, MA} 
%\begin{tightitemize}
%\vspace{\topsep}
%\item Initiated the start up of the group with \textbf{\href{http://faculty.uml.edu/jpropp/}{Prof. James Propp}} and recruited students to join. 
%\item Worked in a small group on combinatorics problems related to the de-randomized Polya and Smirnov boards. 
%\item Presented work at an undergraduate mathematics conference.
%\end{tightitemize}

\sectionspace % Some whitespace after the section

\sectionspace


%----------------------------------------------------------------------------------------

\end{minipage} % The end of the right column

%----------------------------------------------------------------------------------------
%	SECOND PAGE (EXAMPLE)
%----------------------------------------------------------------------------------------

\newpage % Start a new page

\begin{minipage}[t]{0.33\textwidth} % The left column takes up 33% of the text width of the page
\section{Áreas de Interesse}
\vspace{\topsep} % Hacky fix for awkward extra vertical space
\begin{tightitemize}
\item Aeromodelismo
\item Data Science
\item Inteligência Artificial.
\item Otimização
\item Programação
\item Robótica móvel
\item Veículos Autônomos
\item Visão Computacional
\end{tightitemize}

\section{Idiomas}
\vspace{\topsep} % Hacky fix for awkward extra vertical space
\begin{tightitemize}
\item Português (nativo)
\item Inglês (Toefl IBT 106/120)
\end{tightitemize}
\section{GitHub}
\faGithub \href[pdfnewwindow=true]{http://github.com/Williangalvani}{GitHub.com/Williangalvani}

\section{Website}
\faGlobe \href[pdfnewwindow=true]{http://GalvanicLoop.com}{ GalvanicLoop.com}


\end{minipage} % The end of the left column
\hfill
\begin{minipage}[t]{0.66\textwidth} % The right column takes up 66% of the text width of the page


\section{Projetos - VANTs}

\runsubsection{Desenvolvimento/Montagem de VANTs}
\descript{}
\vspace{\topsep} % Hacky fix for awkward extra vertical space
\begin{tightitemize}
 \item Quadcópteros (Multiwii, Apm, KapteinKuk).
 \item Tricóptero (Multiwii). 
 \item Bicóptero (MultiWii).
 \item Asas voadoras.
 \item Aviões.
\end{tightitemize}

\sectionspace % Some whitespace after the section

\runsubsection{ProVant Groundstation} \href[pdfnewwindow=true]{http://github.com/Williangalvani/provant-groundstation}{\faGithub}
\descript{(Python + Qt)}
\begin{tightitemize}
 \item Software de estação base para o ProVant, projeto de Tilt-Rotor da UFSC.
\end{tightitemize}

\sectionspace % Some whitespace after the section

\runsubsection{Rpi Multiwii FPV} \href[pdfnewwindow=true]{http://github.com/Williangalvani/RpiMultiwiiFpv}{\faGithub}
\descript{(Python + Qt)}
\begin{tightitemize}
 \item Experimento para utilizar uma conexão Wi-Fi para telemetria, downlink de vídeo, e uplink de comandos via joystick. Todos os links implementados por UDP para um menor tempo de resposta.

\end{tightitemize}

\sectionspace % Some whitespace after the section

\runsubsection{Drone Token Tracker} \href[pdfnewwindow=true]{http://github.com/Williangalvani/drone-token-tracker}{\faGithub}
\descript{(Python + OpenCV)}
\begin{tightitemize}
 \item Sistema de rastreamento de um token utilizando um quadcoptero no simulador V-Rep.
 \end{tightitemize}
 
\sectionspace % Some whitespace after the section

\runsubsection{DiyOSD MultiWii} \href[pdfnewwindow=true]{http://github.com/Williangalvani/diyosd_multiwii}{\faGithub}
\descript{(Arduino + Baixo nível + Eletrônica)}
\begin{tightitemize}
 \item Projeto de OSD de baixo custo utilizando apenas um arduino e componentes passivos, dependendo fortemente do tempo fixo de execução para manipulação dos dados analógicos.
 \end{tightitemize}
 
\sectionspace % Some whitespace after the section

\runsubsection{Planador Solar Autônomo (WIP) }\href[pdfnewwindow=true]{http://galvanicloop.com/blog/post/13/building-a-solar-powered-glider-wip}{\faExternalLink}\\
\descript{(X-Foil + Impressão 3D + Eletrônica)}
\begin{tightitemize}
 \item Projeto de um moto-planador solar, autônomo tanto em navegação quanto em capacide energética.
 \item Objetivo de identificar e aproveitar termais.
 \end{tightitemize}
 
\sectionspace % Some whitespace after the section

\section{Projetos - Full Stack}

\runsubsection{Website} \href[pdfnewwindow=true]{http://github.com/Williangalvani/meuBlog}{\faGithub}
\descript{(Django + Bootstrap)}
\begin{tightitemize}
 \item Blog pessoal utilizando Django.
\end{tightitemize}

\sectionspace % Some whitespace after the section

\runsubsection{HobbyKing Search} \href[pdfnewwindow=true]{http://github.com/Williangalvani/hkSearcher}{\faGithub}\\
\descript{(Django + Bootstrap + JQuery + JQueryUI)}
\begin{tightitemize}
 \item Sistema alternativo para buscas de componentes no site da HobbyKing, utilizando sliders para filtragem de vários parâmetros.
 \end{tightitemize}
 
\sectionspace % Some whitespace after the section

\runsubsection{Controle de Acesso - DAS}
\descript{(Django + Bootstrap + JQuery)}
\begin{tightitemize}
 \item Sistema de controle de acesso para o Departamento de Automação e Sistemas da UFSC. Este projeto envolveu a engenharia reversa dos equipamentos, software, e banco de dados adquiridos com o equipamento.
 \end{tightitemize}
 
\sectionspace % Some whitespace after the section


\section{Grupos de Competição}

\sectionspace % Some whitespace after the section


\runsubsection{Robota -UFSC}
\descript{\\ Mobile robotics competition team.}
\location{March 2016 – Atualmente}
\vspace{\topsep} % Hacky fix for awkward extra vertical space
\begin{tightitemize}
 \item Projeto de robôs autônomos para competições.
 \item Desenvolvimento de soluções e ferramentas open-source para aplicações de robótica móvel.
\end{tightitemize}









\end{minipage} % The end of the right column

%----------------------------------------------------------------------------------------

\end{document}